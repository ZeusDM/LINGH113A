\lecture{3}{2021-09-08}{Returning to our mini English grammar}

However, this rule \emph{overgenerates}, i.e., it produces sentences which are considered ungrammatical --- it produces too many sentences. For instance, \ref{ex:dogrun} and \ref{ex:dogsruns} are produced by this rule that we might want to exclude from our grammar.

\ex. *Dog run. \label{ex:dogrun}

\ex. *Dogs runs. \label{ex:dogsruns}

Of course, if we said our entire grammar was produced this rule, that would be a massive \emph{undergeneration}, since there are \emph{many} grammatical sentences of English which aren't generated by it, like \ref{ex:happydogsrun} and \ref{ex:dogsalwaysrun}.

\ex. Happy dogs run. \label{ex:happydogsrun}

\ex. Dogs always run. \label{ex:dogsalwaysrun}

\section{Syntactic primitives}

If we take the rule as our starting point, it's clear that we have two types of changes we need to make: we somehow need to introduce either more specific sub-types of nouns and verbs, to rule out things like \ref{ex:dogrun} and \ref{ex:dogsruns}, and we need to add more rules, to cover things like \ref{ex:happydogsrun} and \ref{ex:dogsalwaysrun}. Then, given more data, we might need to refine those new rules, too. This is a lot of what syntax is: coming up with rules, looking at data to see if the rules generate the data, and if not, changing the rules and starting again! In order to come up with these rules, we need some \emph{syntactic primitives}: basic categories of words that we can all agree on. So let's try it.

Now, upon combining our lists, we've probably got a lot of familiar terms like Nouns, Verbs, Adjectives, Adverbs, Prepositions, Determiners --- this is a solid foundation! We use these terms in syntax often, as they are extremely common across languages and can roughly be described in terms of their distribution in addition to their function. For instance, in English, \emph{nouns} are the types of things that can be pluralized, or the things that can follow a determiner. They often represent `people, places, and things' --- but not always. So rather than a meaning-based category, we want a distribution-based category.

You might have come up with a bunch of other terms that we may or may not end up using for our rules in this class --- things like conjunctions, particles, participles, or interjections. We will certainly talk about them as they become relevant to describing data as we see.

Finally, two noun-type things that we ought to consider specially: \emph{pronouns} and \emph{proper names}. Are these distinct syntactic categories from \emph{noun}? We have special names for them, but do they have an identical distribution to nouns like \emph{cat(s)}, \emph{dogs(s)}? What can we do with these?

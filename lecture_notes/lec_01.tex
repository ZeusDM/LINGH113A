\chapter{Introduction}
\lecture{1}{2021-08-30}{What is Syntax?}
\section{What do we study in syntax?}

Syntax is all about sentence \emph{structure.} Syntax is about \emph{form,} and semantics is about meaning. This has a few implications.

A \emph{sentence} is just a string of words. It doesn't necessarily have a meaning assigned to it. (A \emph{proposition} is a statement that has meaning and can be true or false.) We often represent sentences with written text for ease of analysis, but you can study the syntax of spoken or signed language.

\ex. It is raining today. \label{ex:1.1}

\ref{ex:1.1} is an example of a sentence in English. Most English speakers agree that \ref{ex:1.1} is \emph{grammatical}. That means it follows the rules of English syntax. We haven't talked about what exactly those rules are yet, but if you are an English speaker, you have some intuitions about what is grammatical and what is not. This data --- sentences and whether they are grammatical or not --- is our subject of study as syntacticians. Sometimes we can generate the data ourselves! But sometimes we can't, or shouldn't.

\ex. Íbsaan dháabata autóbusaa jíra. \label{ex:1.2}

\ref{ex:1.2} is a sentence from the language Oromo. It is also grammatical.

Some sentences are \emph{grammatical} --- they have a form that follows the rules --- but considered unacceptable. This is a major source of confusion for our data! Here are some such sentences:

\ex. That person is a real *****.
(Slurs are socially unacceptable, but typically grammatical.)

\ex. \# My toothbrush sneezed.
(Semantically unacceptable!)

Finally, grammar in linguistics is \emph{descriptive}, not \emph{prescriptive}. That means if a sentence would be naturally uttered by a speaker of a language, the sentence is considered grammatical for our data! This includes things like:

\ex. What should I talk about?
(Ending a sentence with a preposition happens all the time in speech.)

\ex. Who is this for?
(Using `who' where you might have been taught to use `whom'.)

Ungrammatical sentences are marked with *.
In-between or `marginally grammatical' sentences are marked with a ?.
Sentences that are semantically odd are marked with \#.

\ex. *Ungrammatical is sentence like this.

Grammaticality is not binary. Speakers of the same language can have different grammars, and therefore different judgments. Sometimes you might not even be sure if a given sentence is grammatical for you or not!
